\documentclass[12pt]{article}
\usepackage{lingmacros}
\usepackage{tree-dvips}
\usepackage{amssymb}
\begin{document}

\subsection*{Daniel Culliver}
\section*{35-Z2-4 Supervěž}
Čtení šachovnice samotné už je $N^2$ operací, takže to je základní časová komplexita

První možnost, která člověku napadne, je převést šachovnici na 2D array 1 a 0.
Poté sečíst řádek a sloupec pro každé místo.
Toto vyjde na $N^2 * 2N$ operací, což dohromady se čtením šachovnice vyjde na
$2N^2 * 2N$, nebo-li $4N^3$, což odpovídá časové komplexity O($N^3$).

Existuje ale optimalizace. Místo sečtení řádku a sloupce pro každý čtereček,
sečteme jenom každý řádek a sloupec a uložím ho do array.
Poté stačí projet Každý čtereček, ale místo 2N operací, to bude jenom 2 operace.
Toto vyjde na $2N^2 * 2$ operací, nebo-li časová komplexita O($N^2$).
Vzhledem k tomu, že to je stejná komplexita jako čtení vstupu,
pravděpodobně se nedá dál optimalizovat časová složitost.

Druhá část komplexity je paměťová. Uložení array šachovnice zabírá $N^2$ místa
a součty rádků a sloupců je dalších 2N. Toto vychází na paměťová komplexita O($N^2$).
Ale my nepotřebujeme array šachovnice potom, co jsme sečetli řádky a sloucpe,
tudíž můžeme ukládání arraye zanedbat a rovnou sečíst řádky a sloucpe při čtení vstupu.
Toto zabírá 2N místa a uložení čísla N pro for loopy můžeme zanedbat.
To vychýzí na paměťovou komplexitu O(N).

Jako závěr jsme došli k časové komplexitě O($N^2$) a paměťová komplexita O(N).


\end{document}