\documentclass[12pt]{article}
\usepackage{lingmacros}
\usepackage{tree-dvips}
\usepackage{amssymb}
\begin{document}

\subsection*{Daniel Culliver}
\section*{35-Z2-4 Supervěž}
Čtení šachovnice samotné už je $N^2$ operací, ale bylo mi sděleno velmi dobrý zdojem (Matúšem), 
že časová komplexita se počítá jako lineární vůči čtení celého vstupu, tedy čtení NxN šachovnice má časovou komplexitu O($n$). 

První možnost, která člověku napadne, je převést šachovnici na 2D array 1 a 0.
Poté sečíst řádek a sloupec pro každé místo.
Toto vyjde na $N^2 * 2N$ operací, což dohromady se čtením šachovnice vyjde na
$2N^2 * 2N$, nebo-li $4N^3$, což odpovídá časové komplexity O($N^3$) nebo-li O($n \log{n}$).

Existuje ale optimalizace. Místo sečtení řádku a sloupce pro každý čtereček,
sečteme jenom každý řádek a sloupec a uložím ho do array.
Poté stačí projet Každý čtereček, ale místo 2N operací, to bude jenom 2 operace.
Toto vyjde na $2N^2 * 2$ operací, nebo-li lineární časová komplexita vůči vstupu O($n$).
Vzhledem k tomu, že to je stejná komplexita jako čtení vstupu,
pravděpodobně se nedá dál optimalizovat časová složitost.

Druhá část komplexity je paměťová. Uložení array šachovnice zabírá $N^2$ místa
a součty rádků a sloupců je dalších 2N. Toto vychází na paměťová komplexita O($N^2$).
Ale my nepotřebujeme array šachovnice potom, co jsme sečetli řádky a sloucpe,
tudíž můžeme ukládání arraye zanedbat a rovnou sečíst řádky a sloucpe při čtení vstupu.
Toto zabírá 2N místa a uložení čísla N pro for loopy můžeme zanedbat.
To vychýzí na paměťovou komplexitu O(N), tedy O($\log{n}$).

Jako závěr jsme došli k časové komplexitě O($n$) a paměťová komplexita O($\log{n}$).
\\\\
Omlouvám se za veškeré \emph{ť} v dokumentu. Latex píšu dneska podruhé a pravděpodobně je to nastaveným fontem (ještě reším jak to zpravit).

\end{document}