\documentclass{../../../../ksp}
\usepackage{fancyvrb}

\title{KSP 36-S-1}
\author{Daniel Culliver}
\date{Listopad 2023}

\begin{document}

\maketitle

\section*{Postup k řešení}

Nejdřív si napíšu oběcný vstup, aby bylo jasný co znamená která proměnná:

\begin{Verbatim}[commandchars=\\\{\}, codes={\catcode`$=3\catcode`_=8}]
    n f
    $w_1$, $w_2$ \dots $w_f$, $w_{f+1}$
    $x_{1,1}$, $x_{2,1}$ \dots $x_{f,1}$, $x_{f+1,1}$
    $x_{1,2}$, $x_{2,2}$ \dots $x_{f,2}$, $t_{2}$
    \dots
    $x_{1,n}$, $x_{2,n}$ \dots $x_{f,n}$, $t_{n}$
\end{Verbatim}

Víme, že $p_1 = x_{1,1} \cdot w_{1} + x_{1,2} \cdot w_{2} + \dots + x_{1,f} \cdot w_{f} + 1 \cdot w_{f+1}$, tedy:
\begin{gather*}
    \begin{pmatrix}
        x_{1,1} & x_{1,2} & \cdots & x_{1,f} & 1\\
        x_{2,1} & x_{2,2} & \cdots & x_{2,f} & 1\\
        \vdots & \vdots & \ddots & \vdots & \vdots\\
        x_{n,1} & x_{n,2} & \cdots & x_{n,f} & 1\\
    \end{pmatrix}
    \begin{pmatrix}
        w_{1}\\
        w_{2}\\
        \vdots\\
        w_{f+1}\\
    \end{pmatrix}
    =
    \begin{pmatrix}
        p_{1}\\
        p_{2}\\
        \vdots\\
        p_{n}\\
    \end{pmatrix}
\end{gather*}

A podle úprav v zadání víme, že toto ověřuje $w_1$:

\begin{gather*}
    \begin{pmatrix}
        x_{1,1} & x_{2,1} & \cdots & x_{n,1}
    \end{pmatrix}
    \begin{pmatrix}
        p_{1}\\
        p_{2}\\
        \vdots\\
        p_{n}\\
    \end{pmatrix}
    =
    \begin{pmatrix}
        x_{1,1} & x_{2,1} & \cdots & x_{n,1}
    \end{pmatrix}
    \begin{pmatrix}
        t_{1}\\
        t_{2}\\
        \vdots\\
        t_{n}\\
    \end{pmatrix}
\end{gather*}

Všechny featury dohromady:

\begin{gather*}
    \begin{pmatrix}
        x_{1,1} & x_{2,1} & \cdots & x_{n,1}\\
        x_{1,2} & x_{2,2} & \cdots & x_{n,2}\\
        \vdots & \vdots & \ddots & \vdots\\
        x_{1,f} & x_{2,f} & \cdots & x_{n,f}\\
    \end{pmatrix}
    \begin{pmatrix}
        p_{1}\\
        p_{2}\\
        \vdots\\
        p_{n}\\
    \end{pmatrix}
    =
    \begin{pmatrix}
        x_{1,1} & x_{2,1} & \cdots & x_{n,1}\\
        x_{1,2} & x_{2,2} & \cdots & x_{n,2}\\
        \vdots & \vdots & \ddots & \vdots\\
        x_{1,f} & x_{2,f} & \cdots & x_{n,f}\\
    \end{pmatrix}
    \begin{pmatrix}
        t_{1}\\
        t_{2}\\
        \vdots\\
        t_{n}\\
    \end{pmatrix}
\end{gather*}

A teď přichází úvaha, o kterém si nejsem 100\% jistý, a to je, že rovnice výše, by měl ještě
ověřovat poslední váhu, bias, $w_{f+1}$. Toto docílíme jednoduše - přidáme další řádek pod
maticí featur:

\begin{gather*}
    \begin{pmatrix}
        x_{1,1} & x_{2,1} & \cdots & x_{n,1}\\
        x_{1,2} & x_{2,2} & \cdots & x_{n,2}\\
        \vdots & \vdots & \ddots & \vdots\\
        x_{1,f} & x_{2,f} & \cdots & x_{n,f}\\
        1 & 1 & \cdots & 1\\
    \end{pmatrix}
    \begin{pmatrix}
        p_{1}\\
        p_{2}\\
        \vdots\\
        p_{n}\\
    \end{pmatrix}
    =
    \begin{pmatrix}
        x_{1,1} & x_{2,1} & \cdots & x_{n,1}\\
        x_{1,2} & x_{2,2} & \cdots & x_{n,2}\\
        \vdots & \vdots & \ddots & \vdots\\
        x_{1,f} & x_{2,f} & \cdots & x_{n,f}\\
        1 & 1 & \cdots & 1\\
    \end{pmatrix}
    \begin{pmatrix}
        t_{1}\\
        t_{2}\\
        \vdots\\
        t_{n}\\
    \end{pmatrix}
\end{gather*}

Můžeme teď nahlédnout, že $p_1 + p_2 + \ldots + p_n = t_1 + t_2 + \ldots + t_n$.
Sice toto přímo neznamená, že $p_1 = t_1$ a $p_2 = t_2$, ale logicky dává smysl, že dokonalá predikce bude
rovna opravdové hodnotě.

Toto znamená, že stačí ověřit:

\begin{gather*}
    \begin{pmatrix}
        x_{1,1} & x_{1,2} & \cdots & x_{1,f} & 1\\
        x_{2,1} & x_{2,2} & \cdots & x_{2,f} & 1\\
        \vdots & \vdots & \ddots & \vdots & \vdots\\
        x_{n,1} & x_{n,2} & \cdots & x_{n,f} & 1\\
    \end{pmatrix}
    \begin{pmatrix}
        w_{1}\\
        w_{2}\\
        \vdots\\
        w_{f+1}\\
    \end{pmatrix}
    =
    \begin{pmatrix}
        t_{1}\\
        t_{2}\\
        \vdots\\
        t_{n}\\
    \end{pmatrix}
\end{gather*}

Teď, abychom se dozvědělí optimální váhy $w_1$ až $w_{f+1}$, stačí vynásobit obě strany rovnice zleva
inverzní maticí featur a jako finální rovnici získáme:

\begin{gather*}
    \begin{pmatrix}
        w_{1}\\
        w_{2}\\
        \vdots\\
        w_{f+1}\\
    \end{pmatrix}
    =
    \begin{pmatrix}
        x_{1,1} & x_{1,2} & \cdots & x_{1,f} & 1\\
        x_{2,1} & x_{2,2} & \cdots & x_{2,f} & 1\\
        \vdots & \vdots & \ddots & \vdots & \vdots\\
        x_{n,1} & x_{n,2} & \cdots & x_{n,f} & 1\\
    \end{pmatrix}
    ^{-1}
    \begin{pmatrix}
        t_{1}\\
        t_{2}\\
        \vdots\\
        t_{n}\\
    \end{pmatrix}
\end{gather*}

\end{document}