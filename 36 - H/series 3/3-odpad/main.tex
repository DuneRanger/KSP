\documentclass{../../../ksp}

\title{KSP 36-H-3}
\author{Daniel Culliver}
\date{Leden 2024}

\begin{document}

\maketitle

\section*{Základní myšlenky}

Jako úplně první si stanovíme, že budeme veškeré dělení hromad zaokrouhlovat nahoru. Například u čísla 35, při dělení do dvou hromad bychom doopravdy získali hromady s výškou 17 a 18, ale pro veškeré naše účely můžeme říct, že máme dvakrát 18.
Toto si můžeme dovolit právě proto, že rozdělení hromady s výškou 18 nám zabere jedno operace a stále nám zbyde hromada s výškou 17, tedy celkově provedeme 18 instrukcí.
Kdybychom chtěli rozdělit obě hromady, tak získáme 8, 9, 9, 9, a narážíme opět na stejnou situace, kde je nám vlastně jedno, že jedna hromada je nižší, protože máme další, vyšší hromady.
Tímto si můžeme dovolit veškeré dělení zaokrouhlovat nahoru.
A to, že vůbec dělíme hromady na hromady o stejné velikosti bereme jako samozřejmost.

V předchozím odstavci jsme si vlastně také ukázali, že když máme hromady o výšce $n$ a $n+1$, tak nemá smysl dělit jenom jednu z těch hromad. 
Toto můžeme dále rozšířit na případ, kdy máme $x$ krát výšku $n$, tak je výhodné rozdělit tyto hromady právě tehdy, kdy nejvyšší hromada menší než $n$ má výšku menší (nebo rovno) $n-x$.
Ve spojení s předchozím nálezem, můžeme rovnou seskupit všechny množiny právě $x$ čísel, které se nachází v intervalu $n \pm x$.

Jako další si ukážeme, že výsledný počet operací je $\Omega (\sqrt{N})$, $\BigO (N)$ (počítáme s tím, že alespoň jedna hromada má maximální možnou výšku).
Toto si jednoduše vysvětlíme pomocí dvou extrémních případů.

V prvním případě budeme mít $N-1$ hromad o výšce $1$ a poté jednu hromadu o výšce $N$.
Jediné co nás zajímá je dělení poslední hromady, a je zřejmé, že nejvýhodnější je dělení jeho odmocninou, tedy $\sqrt{N}$.
Je nám jedno, jestli tuto odmocninu zaokrouhlíme nahoru či dolu, v obou případech se nám bude výsledek lišit pouze o jedna.
Případné další zvětšení dělení nám také zlepší výsledek pouze o 1, jestli vůbec.
Pro snažší pochopení si můžeme představit číselnou osu, kde náš dělitel začne vlevo u jedničky a výsledek vpravo u $N$.
Můžeme sledovat, že posunutím dělitele blíž k $\sqrt{N}$ se ná musí o dost víc blížit výsledek k $\sqrt{N}$, protože musí putovat mnohem větší vzdálenost.
Přímo v $\sqrt{N}$ se setkají (protože $\sqrt{N}^2 = N$ pro kladný $N$), a dalším posunutím dělitele doprava se obrátí sitauce, tedy to už není výhodný.

V druhém případě budeme mít $N$ hromad o výšce $N$.
Je zřejmé, že aby jakékoliv dělení bylo výhodné, je potřeba ho provést na všechny hromady, tedy $N$ krát.
Ale pomocí $N$ operací jsme rovnou schopni všechny hromady redukovat na nulu.
V zadání je zmíněno, že hromada má výšku \emph{řádově} $N$, ne přímo N, ale na asymptotickou složitost výsledku to nic nemění.

\section*{Algoritmus}

Nejdřív si uvědomíme, že algoritmus nikdy nepřekoná $\BigO(N)$, protože už jenom procházení a určení nejvyšší hromady trvá $\BigO(N)$.
Zároveň také existují vstupy, které nelze řešit rychleji než $\BigO(N)$ operací.
S touto myšlenkou jsme schopni ``bezpečně'' setřídit naše hromádky pomocí přehrádkové třízení a zachovat tím lineárnost řešení.
Zbytek algoritmu není příliš složitý: začneme od největší hromady a zkontrolujeme, jestli je výhodný zvětšit jeho dělitel o jeden.
Jestli ano, tak to provedeme, ale jestli ne, tak to znamená, že zvětšení dělitele této hromady je výhodný pouze tehdy, kdy zmenšíme dělitel i další hromady. Toto nám umožní ty dvě hromady seskupit (zůstane nejvyšší z nich, ale zvětší se počet potřebných operací pro zvětšení dělitele) a pokračovat dál.

Proces kontroly, jestli je výhodný zvětšit dělitele nějaké hromady, funguje na bází porovnání s maximem, ale naše hromádky už nebudou setřízené už od první dělení hromady.
Jak tedy budeme pamatovat jaké hromady mohou být nejvyšší?
Po prvním dělení máme 2 možnosti, druhou nejvyšší hromadu nebo nově vzniklou hromadu, a rovnou si je oba hodíme do max---heapu.
Už je jasné, že si všechny maxima budeme ukládat do haldy, ale abychom zachovali lineárnost řešení, tak je potřeba ukázat, že naše halda nebude mít velikost $\BigO(N)$.

Začneme myšlenkou, že když máme několik čísel se stejným dělitelem, tak největší z nich je maximem z té skupiny, tedy budeme pamatovat pouze jeho ze všech čísel se stejným dělitelem.
Jinými slovy, v haldě bude maximálně tolik čísel, jaký je největší dělitel, a největší dělitel jsme už stanovili jako $\sqrt{N}$.
Z toho už je jasné omezení velikosti haldy a vložení a smazání nastane ve stejném okamžiku, tedy $\sqrt{N}$ krát.
Tedy veškeré naše operace s haldou se nám vejde do $\BigO(\sqrt{N} \cdot \log(\sqrt{N}))$.

V rámci práci s haldou provedeme veškeré minimalizování nejvyšších hromady a zbyde nám jenom process házení kyseliny, dokud nám nezmizí všechny hromady.
Tento krok můžeme vlastně provést v konstatním čase, ale počet operací ve vstuput nám nakonec vyjde (jak jsme stanovili na začátku) někde v intervalu $\langle \sqrt{N}; N\rangle$.

Celková časová komplexita češení je tedy $\BigO(N)$ a prostorová komplexita (když nepočítáme vstup a výstup) je $\BigO(\sqrt{N})$.

\end{document}