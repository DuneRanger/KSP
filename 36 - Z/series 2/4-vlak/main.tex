% \documentclass{ksp}
\documentclass{../../../ksp}

\title{KSP 36--Z2--4}
\author{Daniel Culliver}
\date{Listopad 2023}

\begin{document}

\maketitle

\section*{Řešení}

Rovnou můžeme říct, že jestliže je na cíli zrychlovák, vstup není možné vyřešit.
Beru to tak, že i když vypnu zrychlovák na posledním místě, tak bych se vlastně nedostal na poslední
místo, tedy nedorazil bych do cíle.
Jestliže tento předpoklad je chybný, i tak bude fungovat můj algoritmus, ale nebude vracet tento edge-case
jako neřešitelný.

Po tomhle víme, že u každého vstupu budeme moct zpomalit na nulovou rychlost u cíle.
Jako další krok projdeme celý vstup a budeme průbežně počítat naši rychlost.
V případě, že kdykoliv bude záporná, zapíšeme si, vypneme zpomalovák, přes které jsem zrovna jeli,
upravíme naši rychlost zpátky na to, co bylo před zpomalovákem a pokračujeme dál.
Je zřejmé, že každé toto vypnutí je nutné, abychom vůbec dojeli do cíle.

Teď víme, že dojedeme do cíle buď s nulovou rychlostí, kdy nemusíme nic dál řešit, nebo s kladnou rychlostí.
V případě, kdy dojedeme do cíle s kladnou rychlostí $v$, tak je zřejmé, že musíme vypnout posledních $v$ zrychlováků.


\end{document}