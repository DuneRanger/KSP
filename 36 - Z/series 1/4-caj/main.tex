% \documentclass{ksp}
\documentclass{../../../ksp-smallText}

\title{KSP 36--Z1--4}
\author{Daniel Culliver}
\date{Srpen 2023}

\begin{document}

\maketitle

\section*{Řešení}

Úloha je poměrně jednoznačná, takže se pustím rovnou do řešení. Budu uvažovat o hmotnosti a zajímavosti pytlíku
jako o dvou různých polích, kde stejné indexy budou odkazovat na vlastnosti téhož pytlíku. Toto nezmění prostorovou komplexitu
a přijde mi to lepší pro znázornění v pseudokódu.

Tato úloha je dost podobná klasické úloze o maximálním součtu. Jen, místo komplikace se zápornými čísly tu máme komplikaci s
omezenou délkou úseku. Ze zadání je zřejmé, že jestliže máme nějaký úsek provazu a naše síla nám umožní vzít ještě jeden pytlík,
vezmeme ho. Tedy, nemusíme řešit všechny možné úseky a jejich zajímavosti, ale jen všechny nejdelší možné úseky, 
které jsme schopni unést a najít nejzajímavější z nich.
 
Jak tyto nejdelší možné úseky najdeme? Velmi jednoduše, začneme z prvního pytlíku a nabereme další pytíky, dokud je už neudržíme.
Takhle získáme první úsek. Abychom mohli nabrat další pytlík, musíme se zbavit nějakých, které už máme. Protože pracujeme
v souvislých úsecích, musíme opusit první pytlík, který jsme nabrali. Jestli už můžeme nabrat další pytlík, tak můžeme v algoritmu pokračovat dál.
V případě, že ne, tak opustíme opět první pytlík z úseku. Toto můžeme zopakovat, dokud nenabereme poslední pytlík na celém provazu,
což znamená, že jsme získali poslední nejdelší možný úsek.

Nejzajímavější z těchto úseků najdeme jednoduše při porovnání pokaždé, když nabereme nový pytlík. Podrobnější ukázka algoritmu
si ukážeme ve formě pseudokódu:

\begin{algorithm*}
    \caption{$largestLimitedSum(H, Z, M)$}
    \begin{algorithmic}
        \Require{array $H$ of weights, array $Z$ of interests, maximum weight $M$}
            \State{$i \gets 0$}
            \State{$j \gets 0$}
            \State{$curInterest \gets Z_0$}
            \State{$maxInterest \gets Z_0$}
            \State{$curWeight \gets H_0$}
            \State{$bestSumInd \gets (i, j)$}
            \While{$j < len(H)$}
                \State{$j$ += 1}
                \State{$curWeight$ += $H_j$}
                \State{$curInterest$ += $Z_j$}
                \While{$curWeight > M$}
                    \State{$curWeight$ -= $H_i$}
                    \State{$curInterest$ -= $Z_i$}
                    \State{$i$ += 1}
                \EndWhile{}
                \If{$curInterest$ > $maxInterest$}
                    \State{$maxInterest \gets curInterest$}
                    \State{$bestSumInd \gets (i, j)$}
                \EndIf{}
            \EndWhile{}
        \State{\Return{$bestSumInd$}}
        \Ensure{pair $bestSumInd$, containing the start and end of the best sum}
    \end{algorithmic}
\end{algorithm*}

Pseudokód byl napsaný s předpokladem, že vstup je už uložený. V takovém případě nemůžeme dosáhout lépší
prostorové složitosti než $\BigO(n)$, čehož jsme dosáhli. I kdybychom neměli vstup uložený a museli bychom ho ukládat,
tak bychom museli jenom ukládat úsek, který nás zrovna zajímá. Tento způsob by sice zabral trochu méně prostoru a
zároveň by celková komplexita zůstala lineární. Jinak jsme také dosáhli časové složitosti $\BigO(n)$,
protože na každý pytlík se koukneme maximálně dvakrát --- jednou při posunu $j$ a podruhé při posunu $i$.
Lepší než lineární časovou složitost také nedokážeme, protože přeci jen musíme alespoň jednou ověřit zajímavost každého prvku.

\end{document}