% \documentclass{ksp}
\documentclass{../../../ksp}

\title{KSP 35-2-S}
\author{Daniel Culliver}
\date{Prosinec 2022}

\begin{document}

\maketitle

\section*{Úkol 1 – Nelineární zobrazení [2b]}
Hned nám napovídá slovo \emph{nelineární}. Příklad funkce, které není lineární je kvadratická funkce.
Jak by fungovalo kvadratické zobrazení? Uveďme nejjednodušší možnou kvadratickou funkci:
\begin{equation*}
    f(\mathbf{x}) = \mathbf{x}^2
\end{equation*}
Tato funkce vrátí vektor, který je transformovaný podle své vlastní velikosti, tedy místo rovnice
$f(a\mathbf{x}) = af(\mathbf{x})$ by platila rovnice $f(a\mathbf{x}) = a^2f(\mathbf{x})$.

\section*{Úkol 2 – Fibonacciho čísla [5b]}
Uznávám, že řešení tohoto problému jsem už viděl a to v knížce \textbf{Průvodce albyrintem algoritmů},
ale protože jsem při čtení zadání problém hned spojil s tím co jsem četl, pokládám to za něco, co už je součástí mých znalostí.

\noindent
Začneme tedy hledáním matice \textbf{Q}, které při vynásobení s maticí
$\begin{pmatrix}
    F_n\\
    F_{n+1}
\end{pmatrix}$
získáme matici
$\begin{pmatrix}
    F_{n+1}\\
    F_n + F_{n+1}
\end{pmatrix}$
Odpověď je hned jasná, dokud si pamatujeme pravidla násobení matic:
\begin{equation*}
    \mathbf{Q} =
    \begin{pmatrix}
        0 & 1\\
        1 & 1
    \end{pmatrix}
\end{equation*}

Teď zbývá za úkol jen provéest násobení v lépe než lineárním čase.
K tomu využijeme vlastnost asociativity u matic a chytrý rekurzivní algoritmus, který dokáže umocnit čísla v času $\BigO(\log{n})$.
Nejprve si ukážeme, že matici
$\begin{pmatrix}
    F_n\\
    F_{n+1}
\end{pmatrix}$
můžeme napsat jako
$ \mathbf{Q}^n
\begin{pmatrix}
    F_0\\
    F_{1}
\end{pmatrix}$.
Můžeme si všimnout, že pro n = 1 už víme, že
$ \mathbf{Q}^n
\begin{pmatrix}
    F_0\\
    F_{1}
\end{pmatrix}$
nám dá
$\begin{pmatrix}
    F_1\\
    F_{2}
\end{pmatrix}$.
Dále také víme, že abychom získali $F_3$, musíme tuto matici vynásobit zleva maticí \textbf{Q}, tedy:
$ \mathbf{Q}\mathbf{Q}
\begin{pmatrix}
    F_0\\
    F_{1}
\end{pmatrix}
=
\begin{pmatrix}
    F_2\\
    F_{3}
\end{pmatrix}$.
A kvůli vlastnosti asociativity, toto můžeme zapsat jako:
$ \mathbf{Q}^2
\begin{pmatrix}
    F_0\\
    F_{1}
\end{pmatrix}
=
\begin{pmatrix}
    F_2\\
    F_{3}
\end{pmatrix}$.
Tím jsme dokázali, že
$ \mathbf{Q}^n
\begin{pmatrix}
    F_0\\
    F_{1}
\end{pmatrix}
=
\begin{pmatrix}
    F_n\\
    F_{n+1}
\end{pmatrix}$.

\indent

Teď stačí jen vysvětlit trik rychlé umocňování.

Nejdříve uvažme: $x\in\mathbb{R}$, kterou umocníme číslem $n\in\mathbb{N}$.
Víme, že $x^{2k} = (x^{2})^{k}$ a $x^{k+1} = x\cdot (x^{k})$, takže přepíšeme sudé $n$ jako $2k$ a liché jako $2k+1$.
Takhle můžeme neustále redukovat exponent, dokud nezísakáme 0 a vrátíme výsledek umocnění $x^0$, tedy 1.
Takhle vypadá algoritmus v pseudokódu:

\begin{algorithm*}
    \caption{FastExponent}
    \begin{algorithmic}
        \Require{$x\in\mathbb{R}$, $n\in\mathbb{N}$}             
            \If{n = 0}
                \State return 1
            \EndIf
            \State $a \gets FastExponent(x, \lfloor n/2 \rfloor)$
            \If{n is even}
                \State return $a \cdot a$
            \Else
                \State return $a \cdot a \cdot x$
            \EndIf
        \Ensure{$x^n$}
    \end{algorithmic}
\end{algorithm*}

Takhle provedeme jen nějakých $\BigO(\log n)$ operací místo $\BigO(N)$.

\section*{Úkol 3 – Dosažitelnost [5b]}

K tomuto úkolu můžeme využít algoritmus z předchozího cvičení.
Umocnění matici cest provedeme přes algoritmus \emph{FastExponent}.
Vzhledem k tomu, že zadání říká, abychom se vyhnuly počítání obrovských čísel, nemůžeme jen tak upravit výslednou matici.
Tudíž musíme jinak.

Protože nás zajímá jen jestli existuje nějaká cesta a ne kolik jich je,
můžeme přidat jednu \emph{for} cyklus, kde každý index matice, který je větší než jedna, změníme na jedničku.
Takhle potom bude nejvetší číslo v matici \emph{m}, tedy počet vrcholů v graphu.
Samozřejmě toto není perfektní řešení, protože pro každé násobení projedeme celou matici.
Tedy provedeme $\BigO(m^2 \cdot \log n )$. Toto ze začátku vypadá dost nevýhodně,
ale to je také protože jsme zanedbali další důležitou změnu v algoritmu a to je, že násobíme matice, ne čísla.
Násobení matice je $m^3$ operací, tudíž můžeme zanedbat $m^2$ jako člen který roste pomaleji.
Tudíž bychom skončili s $\BigO(m^3 \cdot \log n)$, což je stejné jako normální umocnění matice
a skončili bychom s maticí, kde bude 1, jestli existuje cesta z vrcholu za n kroků.

\section*{Úkol 4 – Obecná inverze [3b]}

Je zřejmé, že zadání naznačuje, abychom nejdříve představili zobrazení matici \textbf{A}
a z toho vyvodit její inverzi. Ale vzhledem k tomu, že matice \textbf{A} má velikost 2$\times$2,
je mnohem rycheljší vypočítat inverzní matici $\mathbf{A}^{-1}$ přes determinant a pochopit, proč to tak je.

\begin{equation*}
    \begin{aligned}
        \mathbf{A} & = 
        \begin{pmatrix}
            c & -c\\
            c & c
        \end{pmatrix}\\
        \mathbf{A}^{-1} & = \dfrac{1}{\mathbf{det|A|}}
        \begin{pmatrix}
            c & c\\
            -c & c
        \end{pmatrix}\\
        \mathbf{A}^{-1} & = \dfrac{1}{2c^2}
        \begin{pmatrix}
            c & c\\
            -c & c
        \end{pmatrix}\\
        \mathbf{A}^{-1} & =
        \begin{pmatrix}
            \frac{1}{2c} & \frac{1}{2c}\\
            \frac{-1}{2c} & \frac{1}{2c}
        \end{pmatrix}\\
    \end{aligned}
\end{equation*}

Teď když vidíme původní matici \textbf{A} a její inverze $\mathbf{A^{-1}}$, je už zřejmé,
jak bychom mohli najít inverzi bez typického výpočtu. Můžeme nahlédnout, že původní matice \textbf{A}
je dost podobná k otočení, akorát že místo nul má další \emph{c},
tudíž pro její inverzi je potřeba se těchto \emph{c} zbavit. To zvládneme právě pomocí dělením
extra \emph{c}. Proto je inverzní matice plná $\dfrac{1}{2c}$. Změna pozice mínusu je také zřejmé.


\end{document}