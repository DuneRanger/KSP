\documentclass{../../../ksp}

\title{KSP 35-4-S}
\author{Daniel Culliver}
\date{Březen 2023}

\begin{document}

\maketitle

\section*{Úkol 1 – Mnohoúhelník [4b]}

Přímku v rovině jsme schopni zapsat jako lineární rovnici $y = ax + b$.
Začneme bodem $A = {(a,\ b)}^T$ a směrovým vektorem přímky $\bm{u} = (u_x,\ u_y)$.
Víme, že můžeme vyjádřit přímky jako:
\begin{align*}
    x &= a + u_x \cdot t\\
    y &= b + u_y \cdot t
\end{align*}
Kde $t$ je reálný parametr.
Z první rovnice můžeme vyjádřit $t$: $t = \frac{x-a}{u_x}$ a to můžeme dosadit do druhé rovnice:

\begin{align*}
    y &= b + \frac{x-a}{u_x} \cdot u_y\\
    y &= \frac{u_y}{u_x} \cdot x + b - \frac{u_y}{u_x} \cdot a\\
\end{align*}
Můžeme si všimnout, že $\frac{u_y}{u_x}$ je vlastně tangens úhlu mezi $\bm{u}$ a osou $x$,
ale pro náš případ je vhodnější rovnou používat hodnoty vektoru $\bm{u}$.
Můžeme tedy velmi jednoduše z jednoho vektoru a bodu vytvořit rovnici přímky,
u které lze jednoduše (až na jednu výjimku) určit polohu bodu vůči přímce.

Potom stačí pro každý bod $\bm{p}_1,\ \dots ,\ \bm{p}_n$ vytvořit přímku, která vede k následujícímu bodu
a určit jestli je daný bod $\bm{x}$ ``nad'' nebo ``pod'' přimkou.
Nejdřív musíme určit, který z těchto variant vůbec chceme.


\includegraphics{heptagon2.png}

Ukážeme si příklad sedmiúhelníku. Protože víme, že mnohoúhelník bude vždy konvexní, můžeme jednoduše zpozorovat vlastnost,
že pokud x-ová souřadnice prvního bodu je menší než x-ová souřadnice dalšího bodu (např. body $\bm{p}_1$ a $\bm{p}_2$),
budeme chtít, aby bod $\bm{x}$ ležel pod přímkou. Naopak, když x-ová souřadnice prvního bodu bude větší (např. body $\bm{p}_6$ a $\bm{p}_7$),
budeme chtít, aby bod $\bm{x}$ ležel nad přímkou.
Určení polohy $\bm{x}$ vůči přímce lze jednoduše nerovnicí $y \leq \frac{u_y}{u_x} \cdot x + b - \frac{u_y}{u_x} \cdot a$,
kde můžeme zaměnit $\leq$ za $\geq$, dle potřeby.

Musíme ale ještě vyřešit jednu zvláštní výjimku, kde $u_x = 0$ a nerovnice nebude definovaná (např. pro pravidelný šestiúhleník). Pro tento případ stačí jen
porovnat x-ovou souřadnici bodu $\bm{x}$ s x-ovou souřadnicí libovolného bodu té přímky a získame odpověď. Samozřejmě, je potřeba ještě rozlišit
levou a pravou stranu mnohoúhelníku, ale to můžeme jednoduše nahlédnutím na předchozí, popřípadě dalšímu bodu a opět porovnat x-ové souřadnice.

Sice to bylo dlouhé vysvětlení, ale pro každý bod provedeme konstantní množství operací (počítání vektoru, dosazení do rovnice a rozhodovací logika)
a algoritmus nám vyjde jako lineární.

Napadlo mě také aproximace kružnicí, ale to také vychází lineárně, kvůli počítání středu mnohoúhelník. Proto si nemyslím, že to jde lépe než lineárně.


\section*{Úkol 2 – Vzdálenost od roviny [6b]}


Před tím než začneme řešit úlohu s čísly, nejdřív vyjádříme potřebný vzorec obecně.
Z dočtených informací víme, že v rovině můžeme pracovat s bodem $\bm{x}$ vůči přímce jako v bázi
definovaná vektorem té přímky a vektor na ní kolmou. Tedy
$\bm{x} = \frac{\langle \bm{x}, \bm{u} \rangle}{||u||}\bm{\bm{u}} + \frac{\langle \bm{x}, \bm{n} \rangle}{||\bm{n}||}\bm{n}$.
Také jsme se dočetli o tom, že projekci bodu $\bm{x}$ na přímce lze zapsat jako
$\bm{x}_u = \frac{\langle \bm{x}, \bm{u} \rangle}{||u||}\bm{u}$
A vzdálenost bodu $\bm{x}$ od přímky je rovna vzdálenosti bodu od její projekci.

Toto vše nám stačí, abychom spočítali vzdálenost bodu od roviny v prostoru.

Rovinu $\rho$ máme definovanou jako $\bm{c} + a_1\bm{u}_1 + a_2\bm{u}_2$.
Už rovnou můžemem vyjádřit projekci bodu $\bm{x}$ v rovině $\rho$:
$\bm{x}_{\rho} = \frac{\langle \bm{x}, \bm{u}_1 \rangle}{||\bm{u}_1||}\bm{u}_1 + \frac{\langle \bm{x}, \bm{u}_2 \rangle}{||\bm{u}_2||}\bm{u}_2$
K vyjádření samotného bodu $\bm{x}$ budeme ale potřebovat ještě jeden vektor, který je kolmý na $\bm{u}_1$ a $\bm{u}_2$.
Vzhledem k tomu, že v seriálu se přidává kolmý vektor bez vysvětlení, budu předpokládat, že totéž také můžu udělat.
Máme tedy bázi $\bm{u}_1$, $\bm{u}_2$ a $\bm{n}$. I když vektory $\bm{u}_1$ a $\bm{u}_2$ nejsou na sebe kolmé, za chvíli ukážu proč to není potřeba.
Bod $\bm{x}$ teď můžeme vyjádřit jako
$\bm{x}_{\rho} = \frac{\langle \bm{x}, \bm{u}_1 \rangle}{||\bm{u}_1||}\bm{u}_1 + \frac{\langle \bm{x}, \bm{u}_2 \rangle}{||\bm{u}_2||}\bm{u}_2 + \frac{\langle \bm{x}, \bm{n} \rangle}{||\bm{n}||}\bm{n}$.
Vzdálenost bodu $\bm{x}$ od roviny $\rho$ je tedy rovna $||\bm{x} - \bm{x}_{\rho}||$, což po odečtení vypadá takto:
$||\frac{\langle \bm{x}, \bm{n} \rangle}{||\bm{n}||}\bm{n}||$.

Teď, když víme že nám stačí jen znát vektor kolmý na rovinu od které měříme, můžeme velmi jednoduše vypočítat zadaný příklad.

Mějme tedy $\bm{x} = {(6,\ 3,\ 3)}^T$, $\bm{u}_1 = {(2,\ -3,\ 6)}^T$ a $\bm{u}_2 = {(4,\ 4,\ 2)}^T$.
Kolmý vektor $\bm{n}$ vypočítáme vektorovým násobením: $\bm{n} = \bm{u}_1 \times \bm{u}_2 = {(-30,\ 20,\ 20)}^T$.
Protože tento výpočet dělám manuálně, místo toho abych celý vektor normalizoval, jen rychle zmenším jeho velikost na:
$\bm{n} = {(-3,\ 2,\ 2)}^T$.

Teď můžeme dosadit do naší předchozí rovnice:
\begin{align*}
    ||\bm{x}\rho|| &= ||\frac{\langle \bm{x}, \bm{n} \rangle}{||\bm{n}||}\bm{n}||\\
    &= ||\frac{-6}{\sqrt[]{17}}{(-3,\ 2,\ 2)}^T||\\
    &= ||\frac{{(18,\ -12,\ -12)}^T}{\sqrt[]{17}}||\\
    &= \sqrt[]{\frac{{(324,\ 144,\ 144)}^T}{17}}\\
    &= \sqrt[]{\frac{324 + 144 + 144}{17}}\\
    &= \sqrt[]{36}\\
    &= 6\\
\end{align*}


\section*{Úkol 3 – Fyzikální měření [5b]}

Máme tedy nějaké naměřené hodnoty (body), kterým můžeme přiřadit x-ové a y-ové souřadnice
a hledáme nějakou přimku, abychom minimalizovali vzdálenost těchto bodů od přímky.

Když vyjádříme přimku jako $y = ax + b$, poznáme, že každý z těch bodů můžeme dosadit do této rovnice za $x$ a $y$
a potom získame soustavu rovnic o dvou neznámých $a$ a $b$, které pro nás budou na konci značit koeficienty naší hledané přímky.

Abychom mohli už pracovat přímo s něčím z tohot textu a ne jen z hodnot, které tu nejsou napsané, 
nejdřív vypíšu naší nalezenou soustavu rovnic s dosazenými body:
\begin{align*}
    67 &= 0a + b \\
    66 &= 3a + b\\
    63 &= 10a + b\\
    62 &= 14a + b\\
    60 &= 20a + b\\
\end{align*}

Samozřejmě už víme, že tato soustava nemá řešení, ale hledáme takový $a$ a $b$, které vyjádří naší hledanou přímku.
Naštěstí, rovnice na řešení tohoto problému nám už byla představena: $\mathbb{A}^T \mathbb{Ax} = \mathbb{A}^T \mathbb{x}$, 
kde $\mathbb{A}$ pro nás značí koeficienty v tét soustavě rovnic, $\mathbb{x}$ značí hledané neznámé $a$ a $b$ a $\mathbb{b}$ značí
levou stranu soustavy (y-ové hodnoty).

Dosadíme tedy do rovnici a získáme:

\begin{align*}
    \begin{pmatrix}
        1 & 1 & 1 & 1 & 1 \\
        0 & 3 & 10 & 14 & 20 \\
    \end{pmatrix}
    \begin{pmatrix}
        0 & 1 \\
        3 & 1 \\
        10 & 1 \\
        14 & 1 \\
        20 & 1 \\
    \end{pmatrix}
    \begin{pmatrix}
        a \\
        b \\
    \end{pmatrix}
    &=
    \begin{pmatrix}
        1 & 1 & 1 & 1 & 1 \\
        0 & 3 & 10 & 14 & 20 \\
    \end{pmatrix}
    \begin{pmatrix}
        67 \\
        66 \\
        63 \\
        62 \\
        60 \\
    \end{pmatrix}
    \\
    \begin{pmatrix}
        47 & 5 \\
        703 & 47 \\
    \end{pmatrix}
    \begin{pmatrix}
        a \\
        b \\
    \end{pmatrix}
    &=
    \begin{pmatrix}
        318 \\
        2896 \\
    \end{pmatrix}
    \\
    \begin{pmatrix}
        a \\
        b \\
    \end{pmatrix}
    &=
    \frac{-1}{1306}
    \begin{pmatrix}
        47 & -5 \\
        -703 & 47 \\
    \end{pmatrix}
    \begin{pmatrix}
        318 \\
        2896 \\
    \end{pmatrix}
    \\
    \begin{pmatrix}
        a \\
        b \\
    \end{pmatrix}
    &=
    \frac{-1}{1306}
    \begin{pmatrix}
        466 \\
        -87422 \\
    \end{pmatrix}
    \\
    \begin{pmatrix}
        a \\
        b \\
    \end{pmatrix}
    &=
    \begin{pmatrix}
        \frac{-233}{656} \\
        66 \frac{626}{653} \\
    \end{pmatrix}
\end{align*}

Rychlost vypařování tekutého dusíku tedy můžeme aproximovat jako přímku
$y = \frac{-233}{656}x + 66 \frac{626}{653}$.

\end{document}