% \documentclass{ksp}
\documentclass{../../../ksp}

\title{KSP 35-3-S}
\author{Daniel Culliver}
\date{Únor 2023}

\begin{document}

\maketitle

\section*{Úkol 1 – Matice přechodu [7b]}

Známe tedy bázi $B: 1, x, x^2$ a $C: c_1, c_2, c_3$, které jsou součástí prostoru $Q$, což je množinou všech kvadratických funkcí.
Zároveň také známe jak vypadají $\bm{c}_1, \bm{c}_2$ a $\bm{c}_3$ v kanonické bázi:
\begin{equation}
    \begin{split}        
        \bm{c}_1(x) & = \frac{1}{2}x(x-1) = \frac{1}{2}x^2 - \frac{1}{2}x \\
        \bm{c}_2(x) & = -(x+1)(x-1) = -x^2 + 1 \\
        \bm{c}_3(x) & = \frac{1}{2}x(x+1) = \frac{1}{2}x^2 + \frac{1}{2}x
    \end{split}
\end{equation}
Tudíž nejprve začneme jednoduchým převodem těchto vektorů do bázi $B$.

Vektory v bázi $B$ vypadají takto:
\begin{equation}
    \bm{v} =
    \begin{pmatrix}
        c\\
        b\\
        a
    \end{pmatrix}
\end{equation}

Takže bez žádného zbytečného myšlení převedeme vektory  $\bm{c}_1, \bm{c}_2$ a $\bm{c}_3$ do bázi $B$:
\begin{equation}
        [\bm{c}_1]_B =
        \begin{pmatrix}
            0\\
            -\frac{1}{2}\\
            \frac{1}{2}
        \end{pmatrix}
        \qquad
        [\bm{c}_2]_B =
        \begin{pmatrix}
            1\\
            0\\
            -1
        \end{pmatrix}
        \qquad
        [\bm{c}_3]_B =
        \begin{pmatrix}
            0\\
            \frac{1}{2}\\
            \frac{1}{2}
        \end{pmatrix}
\end{equation}

Tudíž transformační matice z báze $C$ do $B$ $_B[id]_C$ stačí jen vypsat:

\begin{equation}
    _B[id]_C = 
    \begin{pmatrix}
        0 & 1 & 0\\
        -\frac{1}{2} & 0 & \frac{1}{2}\\
        \frac{1}{2} & -1 & \frac{1}{2}
    \end{pmatrix}
\end{equation}

Tedy k výpočtu matice $_B[g]_B$ potřebujeme už jenom matici $_C[id]_B$.
Nejdřív stanovíme $\bm{b}_1, \bm{b}_2$ a $\bm{b}_3$:
\begin{equation}
    [\bm{b}_1]_B =
    \begin{pmatrix}
        1\\
        0\\
        0
    \end{pmatrix}
    \qquad
    [\bm{b}_2]_B =
    \begin{pmatrix}
        0\\
        1\\
        0
    \end{pmatrix}
    \qquad
    [\bm{b}_3]_B =
    \begin{pmatrix}
        0\\
        0\\
        1
    \end{pmatrix}
\end{equation}

které jednoduše převodíme do kanonické bázi:

\begin{equation}
    \bm{b}_1 = 1
    \qquad
    \bm{b}_2 = x
    \qquad
    \bm{b}_3 = x^2
\end{equation}

Tentokrát na převod těchto vektorů do bázi $C$ musíme sledovat jejich hodnoty v bodech
$x = -1, x = 0$ a $x = 1$, což opět jednoduše vypočítáme jako:

\begin{equation}
    [\bm{b}_1]_C =
    \begin{pmatrix}
        1\\
        1\\
        1
    \end{pmatrix}
    \qquad
    [\bm{b}_2]_C =
    \begin{pmatrix}
        -1\\
        0\\
        1
    \end{pmatrix}
    \qquad
    [\bm{b}_3]_C =
    \begin{pmatrix}
        1\\
        0\\
        1
    \end{pmatrix}
\end{equation}

Tedy transformační matice z báze $B$ do $C$ $_C[id]_B$ je:

\begin{equation}
    _C[id]_B = 
    \begin{pmatrix}
        1 & -1 & 1\\
        1 & 0 & 0\\
        1 & 1 & 1
    \end{pmatrix}
\end{equation}

Pro výpočet matice $_B[g]_B$ stačí už jen dosadit:

\begin{equation}
    \begin{split}
        _B[g]_B & =\ _B[id]_C\ \cdot\ _C[g]_C\ \cdot\ _C[id]_B \\
        _B[g]_B & = 
        \begin{pmatrix}
            0 & 1 & 0\\
            -\frac{1}{2} & 0 & \frac{1}{2}\\
            \frac{1}{2} & -1 & \frac{1}{2}
        \end{pmatrix}
        \cdot
        \begin{pmatrix}
            0 & 0 & 1\\
            0 & 1 & 0\\
            1 & 0 & 0
        \end{pmatrix}
        \cdot 
        \begin{pmatrix}
            1 & -1 & 1\\
            1 & 0 & 0\\
            1 & 1 & 1
        \end{pmatrix} \\
        _B[g]_B & = 
        \begin{pmatrix}
            0 & 1 & 0\\
            \frac{1}{2} & 0 & -\frac{1}{2}\\
            \frac{1}{2} & -1 & \frac{1}{2}
        \end{pmatrix}
        \cdot 
        \begin{pmatrix}
            1 & -1 & 1\\
            1 & 0 & 0\\
            1 & 1 & 1
        \end{pmatrix} \\
        _B[g]_B & = 
        \begin{pmatrix}
            1 & 0 & 0\\
            0 & -1 & 0\\
            0 & 0 & 1
        \end{pmatrix} \\
    \end{split}
\end{equation}

Teď si řekneme jak jsme tuto matici mohli vymyslet bez těchto výpočtů.
$g$ je zadefinované jako zobrazení, které zrcadlově převrátí funkci podle osy y.
Toto se napíše jako $g(x) = f(-x)$, což se u kvadratické funkce projeví tak, že z
$ax^2 + bx + c$ stane $ax^2 - bx + c$.
Tedy členy $a$ a $c$ zůstanou stejné a $b$ se vynásobí $-1$, což naše matice $_B[g]_B$,
kterou jsme vypočítali dělá.


\section*{Úkol 2 – Soustavy pomocí maticových prostorů [5b]}

Nejprve máme rozhodnout, kolik řešení má soustava $\bm{Ax}=\bm{b}$.
Tato soustava se dá přepsat jako:
\begin{equation}
    x_1\bm{a}_1 + x_2\bm{a}_2 + \cdots + x_n\bm{a}_n = \bm{b}
\end{equation}
Což se dá napsat jako rozšířenu matici:
\begin{equation}
    \begin{pmatrix}
        \bm{a}_1 & \bm{a}_2 & \cdots & \bm{a}_n & \bigm| & \bm{b}
    \end{pmatrix}
\end{equation}
Kde pravá strana se ve finále bude rovnat $\bm{x}$.

Z tohohle vychází, že aby rovnice $\bm{Ax}=\bm{b}$ měla řešení,
$\bm{b}$ musí být lineární kombinací sloupcových vektorů matice $\bm{A}$

Tudíž v případě že ta rozšířená bude mít řešení, bude mít řešení právě jedno.
To ale neznamená, že i rovnice $\bm{Ax}=\bm{b}$ bude mít také jenom jedno řešení.
Známe totiž bázi ker($\bm{A}$) ze kterého můžeme sestavit celou množinu ker($\bm{A}$).
Protože násobením jakéhokoliv členu množiny ker($\bm{A}$) s maticí $\bm{A}$ získáme nulový vektor,
můžeme tedy přidat jakýkoliv vektor z této množiny k vektoru $\bm{x}$ aniž bychom změnili výsledný vektor.

To tedy vyjde tak, že počet řešení této rovnice je rovný velikostí ker($\bm{A}$), kde nulový vektor $\bm{0}$
můžeme přiřadit k půvdonímu vektoru $\bm{x}$ a ostatní řešení samozřejmě k příslušným lineárním kombinacím, které se nachází v
ker($\bm{A}$).

Ve finále, rovnice $\bm{Ax}=\bm{b}$ buď nemá řešení, má právě jedno (když ker($\bm{A}$) obsahuje pouze $\bm{0}$)
nebo nekonečně mnoho (počet všech lineárních kombinací bázi ker($\bm{A}$)).
Pokud známe jedno řešení $\bm{x}_0$, můžeme množinu řešení zapsat takto:
\begin{equation}
    \{ \bm{x}_0 + \bm{v} \bigm| \bm{Ax} = \bm{b} \land \bm{v} \in ker(\bm{A}) \}
\end{equation}

\section*{Úkol 3 – Rekurence [3b]}

Řešení tohoto problému je velmi podobný řešením Fibonacciho pousloupnosti v minulém sérii.
Můžeme nahlédnout, že $F_{n+2} = F_{n+1} + F_n$ je dost podobné našé nové pousloupnosti
$s_{n+2} = 2s_{n+1} + 3s_n$. Je tedy už zřejmé, že můžeme velmi jednoduše postavit matici, stejně jako pro Fibonacciho řady:

\begin{equation}
    \begin{pmatrix}
        s_{n+1}\\
        s_{n+2}
    \end{pmatrix}
    =
    \begin{pmatrix}
        0 & 1\\
        3 & 2
    \end{pmatrix}
    \begin{pmatrix}
        s_{n}\\
        s_{n+1}
    \end{pmatrix}
\end{equation}
Vyzkoušíme tento vzorec pro $s_0$ a $s_1$ a máme rovnou důkaz indukcí:
\begin{equation}
    \begin{pmatrix}
        0 & 1\\
        3 & 2
    \end{pmatrix}
    \begin{pmatrix}
        s_{0}\\
        s_{1}
    \end{pmatrix}
    =
    \begin{pmatrix}
        0 & 1\\
        3 & 2
    \end{pmatrix}
    \begin{pmatrix}
        0\\
        1
    \end{pmatrix}
    =
    \begin{pmatrix}
        1\\
        2
    \end{pmatrix}
    =
    \begin{pmatrix}
        s_{1}\\
        s_{2}
    \end{pmatrix}
\end{equation}

Tedy:

\begin{equation}
    \begin{pmatrix}
        s_{n-1}\\
        s_{n}
    \end{pmatrix}
    =
    \begin{pmatrix}
        0 & 1\\
        3 & 2
    \end{pmatrix}
    ^n
    \begin{pmatrix}
        s_{0}\\
        s_{1}
    \end{pmatrix}
\end{equation}

Což, jestli chápu ten pojem správně, je explicitním vzorcem této geometrické řady.
Do této vzorce stačí definovat $s_0$ a $s_1$ a následně můžeme vypočítat jakýkoliv člen posloupnosti, která je daná
rekurentním vztahem $s_{n+2} = 2s_{n+1} + 3s_n$.

\end{document}
