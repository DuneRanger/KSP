\documentclass[12pt]{article}
\usepackage{lingmacros}
\usepackage{tree-dvips}
\usepackage{amssymb}
\usepackage{algorithm}
\usepackage{algorithmic}
\usepackage{hyperref}
\usepackage{listings}
\usepackage{color}

\definecolor{dkgreen}{rgb}{0,0.6,0}
\definecolor{gray}{rgb}{0.5,0.5,0.5}
\definecolor{mauve}{rgb}{0.58,0,0.82}

\lstset{frame=tb,
  language=C++,
  aboveskip=3mm,
  belowskip=3mm,
  showstringspaces=false,
  columns=flexible,
  basicstyle={\small\ttfamily},
  numbers=none,
  numberstyle=\tiny\color{gray},
  keywordstyle=\color{blue},
  commentstyle=\color{dkgreen},
  stringstyle=\color{mauve},
  breaklines=true,
  breakatwhitespace=true,
  tabsize=3
}

\author{Daniel Culliver}
\title{35-1-S}

\begin{document}

\maketitle
\section*{Úkol 1 – Robot na Marsu [5b]}
Jsou dány 2 vektory v prostoru $\mathbb{R}^2$: $(x_1,y_1)^T$ a $(x_2,y_2)^T$.
\\
Jestliže se chceme pohybovat v jednom směru, stačí přičíst a odečíst vektory, dokud \emph{x} nebo \emph{y} bude 0.
Toto dokážeme velmi jednoduše. Vezmeme tedy $x_1$ a $x_2$:

\makeatletter
\renewcommand{\algorithmicrequire}{\textbf{Vstup:}}
\renewcommand{\algorithmicensure}{\textbf{Výstup:}}
\makeatother

\begin{algorithm}
    \caption{Vytvoř jednotkový vektor}
    \begin{algorithmic}[1]
        \REQUIRE ($x_1, y_1$), ($x_2, y_2$)
        \STATE $m_1 \gets lcm(x_1, x_2)/x_1$ 
        \STATE $m_2 \gets lcm(x_1, x_2)/x_2$ 
        \STATE $(x, y) \gets m_1*(x_1, y_1)-m_2*(x_2, y_2)$
        \STATE $(x, y) \gets (x, y)/y$
        \ENSURE ($x, y$), kde buď $x$ je 0 a $y$ je buď 1 nebo -1
    \end{algorithmic}
\end{algorithm}

Tento algoritmus je velmi jednoduchý. Zjištění nejmenšího společného násobku je potřeba k tomu,
Následným dělením zjistíme kolikrát musíme $x_1$ a $x_2$ násobit, abychom tohoto násobku docílili.
Tento krok samozřejmě lze optimalizovat, ale vzhledem k tomu, že zadání chce jen popis algoritmu,
nebudu se zaměřovat na komplexitu a efektivita algoritmu, ale jenom na principu.

Když tedy známe $m_1$ a $m_2$, můžeme vektory od sebe odečíst a získame tak vektor, kde $x$ = 0.
Zbývá nám jenom $y$, které může být libovolné číslo. Stačí jen vydělit tento vektor $y$ a máme výlsedný vektor,
který určuje pohyb o 1 metr v jednom směru.

To dělení můžeme také pochopit tak, že všechny kroky s původními vektory vlastně dělíme $y$.
Tedy, robot by spočítal $y$ a následně by se pohyboval $m_1$ krát směrem $(x_1, y_1)/y$ a poté $-m_2$ krát $(x_2, y_2)/y$.

Pro pohyb opačným směrem stačí změnit pořadí vektorů v odečítání a stejný postup můžeme provést pro vynulování $y$ a pohyb ve směru $x$.

Toto je samozřejmě celý s předpokladem, že vectory $(x_1, y_1)$ a $(x_2, y_2)$ nejsou lineárně závislé.
Pro případ, že jsou lineárně závislé, úlohu nelze vyřešit.

\subsection*{Úkol 2 – Nezávislost a soustavy [3b]}

Na řešení této úlohy můžeme odkázat na úkol 1.
Tam jsme uvedli, že úkol 1 má řešení právě tehdy když vectory \textbf{$x_1$} a \textbf{$x_2$} nejsou lineárně závislý.
V takovém případě by totiž vyšel vzniklý vector jako (0, 0).
Stejný princip můžeme uplatnit u soustavy lineárních rovnic.
Tedy, jedna rovnice je lineárně závislá na druhou,
v případě že ji můžeme zcela vynulovat pomocí kroků 1-3 z algoritmu v úkolu 1.

Podobný princip jako z úkolu 1 využívá Gaussova eliminace, která je součástí úkolu 3.

\subsection*{Úkol 3 – Algoritmus eliminace [7b]}

Lehčí varianta (za 4 body) (\href{run:./Gauss-Easy.cpp}{Gauss-Easy.cpp}):
Až po nějaké práci na těžší variantu si uvědomil, že toto řešení funguje jenom pro celočíselné kořeny,
protože všude používám int místo float. Bohužel ale už nemám čas toto opravit, ale jediná potřebná změna která mě napadá je,
že místo $gcm()$ a $lcm()$, které zrovna nefungují s floaty kvůli modula, bych mohl použít ten algoritmus, který jsem popsal v úkolu 1.

Pokus o těžší variantě byl, ale s blížícím školním čtvrtletí jsem musel pokus opustit za upřednostnění školí práce.
\lstinputlisting[breaklines]{Gauss-Easy.cpp}

\end{document}