\documentclass[12pt]{article}
\usepackage{lingmacros}
\usepackage{tree-dvips}
\usepackage{amssymb}
\begin{document}

\subsection*{Daniel Culliver}
\section*{35-1-4 Atlas zlomků}

Každý zlomek v základním tvaru je podíl dvou nesoudělných čísel.
\\
Neboli zlomek $\frac{X}{Y}$, kde X $\nmid$ Y.
\\
Tedy pro každý Y, musíme vygenerovat Y-$d_{\mathbb{D}_Y}$+1 zlomků, 
kde $d_{\mathbb{D}_Y}$ je délka množiny $\mathbb{D}_Y$, která vyjadřuje množinu všech dělitelů čísla Y.
Je potřeba přičíst 1, abychom počítali se zlomkem $\frac{1}{Y}$
\\
\\
Zatím vychází, že pro generování zlomků pro jmenovatele Y je potřeba vypsat Y-$d_{\mathbb{D}_Y}$+1 zlomků.
Ještě musíme započítat hledání všech dělitelů čísla Y, který využívá algoritmus s časovou komplexitou O($Y^{\frac{1}{3}}$).
Protože tato komplexita je třetí mocninou Y, zanedbáme ji a budeme dál počítat jen s množstvím operací Y-$d_{\mathbb{D}_Y}$+1, protože roste téměř lineárně.
\\
Kvůli zanedbatelného růstu množiny $\mathbb{D}_Y$, celé generování zlomků bychom mohli zjednodušit na:
\[
  \sum_{Y=1}^{N} Y = \frac{N(N+1)}{2}
\]
\\
Což vychází jako časová komplexita O($N^{2}$).
\\
Tato časová komplexita vychází i kdybychom iterovali přes všechny možnosti a odstranili nevhodné zlomky.
Tudíž bych očekával, že rychlejší algoritmus existuje, ale nenapadá mě.
\\
\\
Dobrou zprávou je, že paměťová složitost algoritmu je velmi dobrá.
\\
Protože každé Y vyžaduje jenom množinu $\mathbb{D}$, abychom věděli které čitatele přeskočit,
paměťová složitost algoritmu je jenom O(max($d_{\mathbb{D}_Y}$)),
nebo-li největší velikost množiny $\mathbb{D}_Y$ (Nemusí být nutně $\mathbb{D}_N$)
což bych neočekával že by ani dosáhlo řádu stovek (vzhledem k tomu že časová složitost bude limitujícím faktorem)

\end{document}