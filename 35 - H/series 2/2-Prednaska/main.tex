\documentclass{../../../ksp}

\title{KSP 35-H-2-2}
\author{Daniel Culliver}
\date{Prosinec 2022}

\algnewcommand{\LineComment}[1]{\Statex \(\triangleright\) #1}

\begin{document}

\maketitle

\section*{Řešení}

Ze zadání vímě, že musíme pracovat s binární haldou.
Začneme tedy nejjednodušším řešením a to je postupné nalezení a odebírání nejemnšího prvku,
dokud nezískáme K jako kořen haldy. Toto je vlastně (pravděpodobně) ten algoritmus využitý na té přednášce
a v tomto případě by měl časovou komplexitu $\BigO(K \log{N})$.
Protože máme zaručeno, že \emph{K} je řádově menší než \emph{N}, nemusíme řešit případ,
kde by bylo rychlejší hledat od největšího prvku a můžeme se soustředit na hledání nejnižšího.


I přes to, že komplexita $\BigO(K \log{N})$ není vůbec špatná, určitě se dá zlepšít. Hlavně, v rámci
$\BigO$ notaci ztratíme veškeré konstanty, a protože celá komplexita programu pravděpodobně
(zatím předpokládám) bude větší než $\BigO(\log{N})$, většina optimalizací budou v počtu kroků v algoritmu.

Jako první pokus o řešení, zkusím napsat algoritmus, který bude hledat K-tý nejnižší prvek v binárním haldě,
bez toho, abych pokaždé odstranil nejnižší prvek a hledal nový kořen.

\begin{algorithm}
    \caption{Hledání K-tého nejnižší členu}
    \begin{algorithmic}[1]
        \Require{\emph{K}, \emph{heap}}             
            \State $savedVals[] \gets ordered\ array$ \Comment{An array of ordered, saved values}
            \State Insert \emph{heap[1]} into \emph{savedVals} 
            \State $curInd \gets 1$ \Comment{Indexing from 1}
            \LineComment{Refers to the index of the last "stable" value in \emph{savedVals}}
            \While{\emph{K} > \emph{curInd}}
                \If{\emph{savedVals[curInd]} has children nodes}
                    \State \emph{BinaryInsert}(\emph{savedVals},\ start: \emph{curInd+1},\ value: \emph{savedVals[curInd].leftChild})
                    \State \emph{BinaryInsert}(\emph{savedVals},\ start: \emph{curInd+1},\ value: \emph{savedVals[curInd].rightChild})
                    \LineComment{A slightly modified binary search for the correct range to which it can put the value}
                    \LineComment{This will always have $\BigO(\log{N})$ and $\Omega(\log{N})$ complexity}
                \EndIf
                \State $curInd \gets curInd+1$
            \EndWhile
        \Ensure{savedVals[K]}
    \end{algorithmic}
\end{algorithm}

Teď se pokusím algoritmus vysvětlit.

Máme poli \emph{savedVals}, který na začátku jenom kořen haldy.
Víme, že je na správném místě (je to nejmenší prvek na místě 1) a přidáme jeho potomky pomocí upraveného binary search,
který věřím je dost triviálně upravený, že nemusím dokazovat jeho komplexitu.
Protože jsme věděli, že předchozí prvek byl na správném místě, ten binaryInsert mohl začít od náseldujícího indexu.
Tímhle skončí první cyklus algoritmu.

Na následujícím cyklu je nejmenší prvek na indexu o jeden větší, ten máme zase zaručený, že je na správném místě
a opět přidáme jeho potomky pomocí \emph{binaryInsert}.
V případě že nejmenší prvek nemá žádné potomky, tak se jenom zmenší oblast \emph{nestabilních} čísel (t.j. čísla, která nemusí být na správné pozici).
To je pro algoritmus jenom výhodné, takže to zanedbáme při počítání $\BigO$ komplexity.

V případě, že každý prvek v \emph{savedVals} má potomky, na každém kroku přidáme 2 \emph{nestabilní} prvky,
ale jeden předchozí \emph{nestabilní} prvek se stane \emph{stabilní}.
To znamená, že na každém kroku získáme 1 \emph{stabilní} prvek a oblast \emph{nestabilních} prvků se zvýšší o 1.
A protože skončíme první cyklus algoritmu s jedním \emph{stabilním} prvek a dvěma \emph{nestabilními} prvky,
je zřejmé, provedeme \emph{K} cyklů, abychom našli \emph{K}-tý prvek a oblast \emph{nestabilních} prvků bude mít velikost \emph{K+1}
Tedy nejvýše $K$ krát provedeme $2\log{K}$, což vychází jako $\BigO(K \log{K})$.

\end{document}

\templatecredit
