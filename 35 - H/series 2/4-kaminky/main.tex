% \documentclass{ksp}
\documentclass{../../../ksp}

\title{KSP 35-H-2-2}
\author{Daniel Culliver}
\date{Prosinec 2022}

\begin{document}

\maketitle

\section*{Řešení}
Začneme prvním algoritmem, který mi napadl. Začíná už při načítání vstupu a posuptně čte každou barvu
a ukládá ho do pole. V případě že nově přečtená barva a předchozí barva je stejná, odstraníme tu předchozí z pole
a tu novou nepřídáme. Tím jsem vlastně odstranili dvojici vedlejších kamínků.

Poté pokračujeme dál ve čtení vstupu s tím, že poslední barva je (aspoň na další porovnání) před-předchozí barvou.
Tím pádem při jednom projetí vstupu zachytíme všechny dvojice.

V případě že vstup lze vyřešit, časová komplexita bude $\BigO(N)$ a $\Omega(N)$ a prostorová komplexita
bude nejvýše $N/2$, což je vlastně také $\BigO(N)$.
V případě, že vstup nebude možný vyřešit, tak jediné co se změní je maximální prostorová komplexita na $N$,c
ož ale vůbec nezmění $\BigO$.

Hlavní důvod proč tento algoritmus funguje je, protože nezáleží v jakém pořadí odebíráme dvojice kamínků z řady.
Jako důkaz vezmeme první čtyři kamínky stejné barvy, které jsou v zadané řade a označíme je $k_1, k_2, k_3, k_4$.
Uvažme tedy situaci, kde nechceme odstranit první možnou dvojici $k_1$ a $k_2$
a chceme $k_2$ odstranit s jiným kamínkém též barvy. Zbývají nám 2 možnosti, $k_3$ a $k_4$. $k_2$ nemůžeme spojit s $k_4$,
protože vždycky bude překážet $k_3$ v jejich cestě. Takže musíme spojit $k_2$ s $k_3$. V případě že mezi $k_2$ s $k_3$ jsme schopni
odstranit všechny dvojice a nakonec i $k_2$ s $k_3$, musíme poté ještě spojit $k_1$ s $k_4$. Opět, v případě že vstup je řešitelný,
musíme být schopni odstranit všechny dvojice mezi $k_1$ s $k_4$.

V případě že jsme tohohle opravdu schopni, tak to znamená,
že jsme schopni odstranit kamínky mezi $k_1$ a $k_2$ a poté kamínky mezi $k_3$ a $k_4$. Tady už vidíme, že by tedy bylo také možné
odstranit normálně $k_1$ a $k_2$ a poté $k_3$ a $k_4$. Dokonce ani nemusíme být schopni odstranit ty kamínky mezi
$k_2$ a $k_3$. Samozřejmě v takovém případě bychom nemohli vyřešit vstup, ale jen poukazuju na to, že spojit 2 kamínky se stejnou barvou
je jednodušší, když jsou nejbližší.

Ještě ukážu algoritmus v pseudokodu:

\begin{algorithm*}
    \begin{algorithmic}
        \Require{$cols$ - sequence of colours }
        \State $saved[] \gets array of saved colours$
        \For{every $colour$ in $cols$}
            \If{$colour$ is equal to $saved[last value]$}
                \State remove $saved[last value]$ from $saved$
            \Else
                \State add $colour$ to $saved$
            \EndIf
        \EndFor
        \If{$saved$ is empty}
            \State return true
        \Else
            \State return false
        \EndIf
        \Ensure{Whether or not the input is solveable}
    \end{algorithmic}
\end{algorithm*}

Tento algoritmus je i vhodný na programování pomocí arraye, protože odebírá jen prvky od konce arraye,
tudíž se nemusí starat o pře-indexování ostatních prvků.


\end{document}